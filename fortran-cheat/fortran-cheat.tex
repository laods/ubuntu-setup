\documentclass[10pt,a4paper]{article}

% Packages
\usepackage{fancyhdr}           % For header and footer
\usepackage{multicol}           % Allows multicols in tables
\usepackage{tabularx}           % Intelligent column widths
\usepackage{tabulary}           % Used in header and footer
\usepackage{hhline}             % Border under tables
\usepackage{graphicx}           % For images
\usepackage{xcolor}             % For hex colours
%\usepackage[utf8x]{inputenc}    % For unicode character support
\usepackage[T1]{fontenc}        % Without this we get weird character replacements
\usepackage{colortbl}           % For coloured tables
\usepackage{setspace}           % For line height
\usepackage{lastpage}           % Needed for total page number
\usepackage{seqsplit}           % Splits long words.
%\usepackage{opensans}          % Can't make this work so far. Shame. Would be lovely.
\usepackage[normalem]{ulem}     % For underlining links
% Most of the following are not required for the majority
% of cheat sheets but are needed for some symbol support.
\usepackage{amsmath}            % Symbols
\usepackage{MnSymbol}           % Symbols
\usepackage{wasysym}            % Symbols
%\usepackage[english,german,french,spanish,italian]{babel}              % Languages

% Document Info
\author{karlp}
\pdfinfo{
  /Title (fortran-90.pdf)
  /Creator (Cheatography)
  /Author (karlp)
  /Subject (Fortran 90 Cheat Sheet)
}

% Lengths and widths
\addtolength{\textwidth}{6cm}
\addtolength{\textheight}{6cm}
\addtolength{\hoffset}{-3cm}
\addtolength{\voffset}{-2cm}
\setlength{\tabcolsep}{0.2cm} % Space between columns
\setlength{\headsep}{-12pt} % Reduce space between header and content
%\setlength{\headheight}{85pt} % If less, LaTeX automatically increases it
\renewcommand{\footrulewidth}{0pt} % Remove footer line
\renewcommand{\headrulewidth}{0pt} % Remove header line
\renewcommand{\seqinsert}{\ifmmode\allowbreak\else\-\fi} % Hyphens in seqsplit
% This two commands together give roughly
% the right line height in the tables
\renewcommand{\arraystretch}{1.3}
\onehalfspacing

% Commands
\newcommand{\SetRowColor}[1]{\noalign{\gdef\RowColorName{#1}}\rowcolor{\RowColorName}} % Shortcut for row colour
\newcommand{\mymulticolumn}[3]{\multicolumn{#1}{>{\columncolor{\RowColorName}}#2}{#3}} % For coloured multi-cols
\newcolumntype{x}[1]{>{\raggedright}p{#1}} % New column types for ragged-right paragraph columns
\newcommand{\tn}{\tabularnewline} % Required as custom column type in use

% Font and Colours
\definecolor{HeadBackground}{HTML}{333333}
\definecolor{FootBackground}{HTML}{666666}
\definecolor{TextColor}{HTML}{333333}
\definecolor{DarkBackground}{HTML}{518F6C}
\definecolor{LightBackground}{HTML}{F4F8F5}
\renewcommand{\familydefault}{\sfdefault}
\color{TextColor}

% Header and Footer
\pagestyle{fancy}
\fancyhead{} % Set header to blank
\fancyfoot{} % Set footer to blank



\begin{document}
\raggedright
\raggedcolumns

% Set font size to small. Switch to any value
% from this page to resize cheat sheet text:
% www.emerson.emory.edu/services/latex/latex_169.html
\footnotesize % Small font.

\begin{multicols*}{3}

\begin{tabularx}{5.377cm}{X}
\SetRowColor{DarkBackground}
\mymulticolumn{1}{x{5.377cm}}{\bf\textcolor{white}{Boiler plate program}}  \tn
\SetRowColor{LightBackground}
\mymulticolumn{1}{x{5.377cm}}{program name \newline     implicit none \newline     ! module imports go here \newline     ! local variables go here  \newline     ! code goes here, there is not main entry point. \newline end program} \tn 
\hhline{>{\arrayrulecolor{DarkBackground}}-}
\end{tabularx}
\par\addvspace{1.3em}

\begin{tabularx}{5.377cm}{X}
\SetRowColor{DarkBackground}
\mymulticolumn{1}{x{5.377cm}}{\bf\textcolor{white}{Module}}  \tn
\SetRowColor{LightBackground}
\mymulticolumn{1}{x{5.377cm}}{module name \newline     implicit none \newline     use other\_mod ! import another module \newline     ! local variables go here  \newline real, private :: a ! Private variable \newline     contains \newline     ! functions and subroutines goes here \newline end module} \tn 
\hhline{>{\arrayrulecolor{DarkBackground}}-}
\end{tabularx}
\par\addvspace{1.3em}

\begin{tabularx}{5.377cm}{X}
\SetRowColor{DarkBackground}
\mymulticolumn{1}{x{5.377cm}}{\bf\textcolor{white}{Function}}  \tn
\SetRowColor{LightBackground}
\mymulticolumn{1}{x{5.377cm}}{! return type goes before the function \newline integer function f(arg1, arg2, ...) \newline ! declarations, including input args \newline     f = 42  ! the function name is result var \newline end function} \tn 
\hhline{>{\arrayrulecolor{DarkBackground}}-}
\end{tabularx}
\par\addvspace{1.3em}

\begin{tabularx}{5.377cm}{X}
\SetRowColor{DarkBackground}
\mymulticolumn{1}{x{5.377cm}}{\bf\textcolor{white}{Subroutine}}  \tn
\SetRowColor{LightBackground}
\mymulticolumn{1}{x{5.377cm}}{! subroutines does not return a value \newline ! but they have out parameters \newline subroutine sub(a, b, c) \newline     integer, intent(in) :: a \newline     integer, intent(inout) :: b \newline     integer, intent(out) :: c \newline  c = a + b \newline end subroutine} \tn 
\hhline{>{\arrayrulecolor{DarkBackground}}-}
\end{tabularx}
\par\addvspace{1.3em}

\begin{tabularx}{5.377cm}{x{2.28942 cm} x{2.68758 cm} }
\SetRowColor{DarkBackground}
\mymulticolumn{2}{x{5.377cm}}{\bf\textcolor{white}{Data types}}  \tn
% Row 0
\SetRowColor{LightBackground}
`logical` & boolean \tn 
% Row Count 1 (+ 1)
% Row 1
\SetRowColor{white}
`integer(kind=1)` & 8 bits signed integer \tn 
% Row Count 2 (+ 1)
% Row 2
\SetRowColor{LightBackground}
`integer(kind=4)` & 32 bits signed integer \tn 
% Row Count 4 (+ 2)
% Row 3
\SetRowColor{white}
`integer(kind=8)` & 64 bits signed integer \tn 
% Row Count 6 (+ 2)
% Row 4
\SetRowColor{LightBackground}
`real(kind=4)` & simple precision float \tn 
% Row Count 8 (+ 2)
% Row 5
\SetRowColor{white}
`real(kind=8)` & double precision float \tn 
% Row Count 10 (+ 2)
% Row 6
\SetRowColor{LightBackground}
`complex(kind=4)` & simple precision complex number \tn 
% Row Count 12 (+ 2)
% Row 7
\SetRowColor{white}
`complex(kind=8)` & double precision complex number \tn 
% Row Count 14 (+ 2)
% Row 8
\SetRowColor{LightBackground}
\seqsplit{`character(len=n)} & string of size n \tn 
% Row Count 15 (+ 1)
% Row 9
\SetRowColor{white}
`integer, dimension(m, n)` & 2D array of shape (m, n) \tn 
% Row Count 17 (+ 2)
\hhline{>{\arrayrulecolor{DarkBackground}}--}
\end{tabularx}
\par\addvspace{1.3em}

\begin{tabularx}{5.377cm}{X}
\SetRowColor{DarkBackground}
\mymulticolumn{1}{x{5.377cm}}{\bf\textcolor{white}{Loops}}  \tn
\SetRowColor{LightBackground}
\mymulticolumn{1}{x{5.377cm}}{do while (logical expr) \newline    ! Control: \newline    ! exit for break \newline    ! cycle for continue \newline end do \newline  \newline label: do i = start, stop ! , step ! (optional) \newline     ! statements \newline end do \newline \newline where (array logical) \newline ! array operations \newline elsewhere \newline ! rest of array operations \newline end where } \tn 
\hhline{>{\arrayrulecolor{DarkBackground}}-}
\end{tabularx}
\par\addvspace{1.3em}

\begin{tabularx}{5.377cm}{X}
\SetRowColor{DarkBackground}
\mymulticolumn{1}{x{5.377cm}}{\bf\textcolor{white}{Conditions}}  \tn
\SetRowColor{LightBackground}
\mymulticolumn{1}{x{5.377cm}}{if (cond) then \newline     ! ... \newline else if (cond) then \newline     ! ... \newline else \newline     ! ... \newline end if} \tn 
\hhline{>{\arrayrulecolor{DarkBackground}}-}
\end{tabularx}
\par\addvspace{1.3em}

\begin{tabularx}{5.377cm}{X}
\SetRowColor{DarkBackground}
\mymulticolumn{1}{x{5.377cm}}{\bf\textcolor{white}{Derived types (structures)}}  \tn
\SetRowColor{LightBackground}
\mymulticolumn{1}{x{5.377cm}}{type new\_type \newline   integer :: i \newline  ! more fields \newline end type \newline  \newline type(new\_type) :: obj \newline obj\%i = 1 } \tn 
\hhline{>{\arrayrulecolor{DarkBackground}}-}
\end{tabularx}

\par\addvspace{1.3em}
\begin{tabularx}{5.377cm}{x{2.94011 cm} x{2.03689 cm} }
\SetRowColor{DarkBackground}
\mymulticolumn{2}{x{5.377cm}}{\bf\textcolor{white}{Arrays}}  \tn
% Row 0
\SetRowColor{LightBackground}
real, dimension(n,m) :: arr & Declare array \tn 
\SetRowColor{white}
arr(i), arr(i:j:stride) & Access \tn 
\SetRowColor{LightBackground}
arr = (/ a,b,c /) & \tn 
\SetRowColor{white}
rank(a), size(a), size(a,dim), shape(a) & Array size \tn 
\SetRowColor{LightBackground}
dot\_product(vec1,vec2) & Vector product \tn
\SetRowColor{white}
matmul(mat1,mat2) & Matrix mult\tn 
\SetRowColor{LightBackground}
reshape(...), cshift(...), eoshift(...), transpose(...) & Array manipulations \tn
\SetRowColor{white}
real, dimension (:,:), allocatable :: darr & Dynamic array \tn 
\SetRowColor{LightBackground}
allocate(darr) \newline deallocate(darr) & Allocate and deallocate \tn 
\hhline{>{\arrayrulecolor{DarkBackground}}--}
\end{tabularx}
\par\addvspace{1.3em}

\begin{tabularx}{5.377cm}{X}
\SetRowColor{DarkBackground}
\mymulticolumn{1}{x{5.377cm}}{\bf\textcolor{white}{Building}}  \tn
\SetRowColor{LightBackground}
\mymulticolumn{1}{x{5.377cm}}{
\# Compile to object (.o) and modules files (.mod) \newline
gfortran -c testModule.f90 mainProg.f90 \newline
\# Then link into executable \newline
gfortran -o run testModule.o mainProg.o
} \tn 
\hhline{>{\arrayrulecolor{DarkBackground}}-}
\end{tabularx}
\par\addvspace{1.3em}

\begin{tabularx}{5.377cm}{x{2.4885 cm} x{2.4885 cm} }
\SetRowColor{DarkBackground}
\mymulticolumn{2}{x{5.377cm}}{\bf\textcolor{white}{Native functions}}  \tn
% Row 0
\SetRowColor{LightBackground}
abs(v) & absolute value of v \tn 
% Row Count 1 (+ 1)
% Row 1
\SetRowColor{white}
aimag(z) & imaginary part of z (single prec) \tn 
% Row Count 3 (+ 2)
% Row 2
\SetRowColor{LightBackground}
int(v, kind) & truncates toward zero to convert to integer \tn 
% Row Count 6 (+ 3)
% Row 3
\SetRowColor{white}
ceiling(v, kind) & ceiling and convert \tn 
% Row Count 7 (+ 1)
% Row 4
\SetRowColor{LightBackground}
floor(v, kind) & floor and convert \tn 
% Row Count 8 (+ 1)
% Row 5
\SetRowColor{white}
modulo(a, p) & polymorphic modulo (sign of p) \tn 
% Row Count 10 (+ 2)
% Row 6
\SetRowColor{LightBackground}
cmplx(x, y, kind) & make a complex from floats \tn 
% Row Count 12 (+ 2)
% Row 7
\SetRowColor{white}
conjg(z) & conjugate complex \tn 
% Row Count 13 (+ 1)
% Row 8
\SetRowColor{LightBackground}
cos(x), dcos(x), ccos(x) & cosine \tn 
% Row Count 15 (+ 2)
% Row 9
\SetRowColor{white}
sin(x), dsin(x), csin(x) & sine \tn 
% Row Count 17 (+ 2)
% Row 10
\SetRowColor{LightBackground}
acos(x), dacos(x) & inverse cosine \tn 
% Row Count 18 (+ 1)
% Row 11
\SetRowColor{white}
asin(x), dasin(x) & inverse sine \tn 
% Row Count 19 (+ 1)
% Row 12
\SetRowColor{LightBackground}
dprod(x, y) & x * y as a double \tn 
% Row Count 20 (+ 1)
% Row 13
\SetRowColor{white}
exp(x), dexp(x), cexp(x) & exponential \tn 
% Row Count 22 (+ 2)
% Row 14
\SetRowColor{LightBackground}
erf(x), derf(x) & error function \tn 
% Row Count 23 (+ 1)
% Row 15
\SetRowColor{white}
erfc(x), derfc(x) & complementary error function \tn 
% Row Count 25 (+ 2)
% Row 16
\SetRowColor{LightBackground}
hypot(a, b) & hypothenuse of x and y (single prec) \tn 
% Row Count 27 (+ 2)
% Row 17
\SetRowColor{white}
log(x), log10(x) & natural and base 10 logarithm \tn 
% Row Count 29 (+ 2)
% Row 18
\SetRowColor{LightBackground}
max(a, b, ...), min(a, b, ...) & polymorphic extrema \tn 
% Row Count 31 (+ 2)
% Row 19
\SetRowColor{white}
sum(arr), product(arr), minval(arr), maxval(arr), all(arr), any(arr) & polymorphic reduction of array, all can be used with dim as second variable \tn 
% Row Count 4 (+ 4)
% Row 20
\SetRowColor{LightBackground}
minloc(arr, dim), maxloc(arr, dim) & the indices of extrema in along axis dim \tn 
% Row Count 14 (+ 2)
\hhline{>{\arrayrulecolor{DarkBackground}}--}
\end{tabularx}
\par\addvspace{1.3em}

\begin{tabularx}{5.377cm}{x{2.14011 cm} x{2.83689 cm} }
\SetRowColor{DarkBackground}
\mymulticolumn{2}{x{5.377cm}}{\bf\textcolor{white}{Operators}}  \tn
% Row 0
\SetRowColor{LightBackground}
** & exponentiation \tn 
% Row Count 4 (+ 1)
% Row 3
\SetRowColor{white}
== & equality (numbers) \tn 
% Row Count 5 (+ 1)
% Row 4
\SetRowColor{LightBackground}
/= & inequality (numbers) \tn 
% Row Count 6 (+ 1)
% Row 5
\SetRowColor{white}
.eqv. & equality (booleans) \tn 
% Row Count 9 (+ 1)
% Row 8
\SetRowColor{LightBackground}
.neqv. & inequality (booleans) \tn 
% Row Count 10 (+ 1)
% Row 9
\SetRowColor{white}
.and., .or., .not. & logicals \tn 
% Row Count 12 (+ 2)
\hhline{>{\arrayrulecolor{DarkBackground}}--}
\end{tabularx}
\par\addvspace{1.3em}

\begin{tabularx}{5.377cm}{X}
\SetRowColor{DarkBackground}
\mymulticolumn{1}{x{5.377cm}}{\bf\textcolor{white}{Input/Output}}  \tn
\SetRowColor{LightBackground}
\mymulticolumn{1}{x{5.377cm}}{
print *, str1, str2, ... \newline
read *, var1, var2, ... \newline
write(unit\_nr, *) output \newline
open (unit\_nr, file=filename)
} \tn
\hhline{>{\arrayrulecolor{DarkBackground}}-}
\end{tabularx}
\par\addvspace{1.3em}



% That's all folks
\end{multicols*}

\end{document}
